\section{Introduction}
\label{sec:intro}
The problem of \ac{TTE} has been receiving enormous interest over the past few years as it forms the fundamental building block in  applications such as route planning, traffic dispatching, cost estimation and ride dispatching used in navigation engines, ride-hailing, food delivery and logistics. Although \ac{TTE} is a well studied problem over several years, the increased availability of several traffic related data-streams from various sources such as smart phones, vehicles and sensors along with big-data infrastructure (e.g. Apache Kafka) opens the possibilities for collection, management and analysis of vast amounts of GPS probe data. Measuring point-to-point travel time estimates and building \ac{ATMS} could eventually help ease congestion given the constraints in terms of building new infrastructure.



There are several challenges towards accurate \ac{TTE}. Primary among those are noisy GPS data which makes it hard to estimate the exact location of vehicles especially in {\it urban canyons} with tall skyscrapers preventing effective GPS signal reception. Furthermore, in several cities, the map data like those from \ac{OSM} is rather inaccurate with several missing roads. Another challenge with using \ac{FCD} for \ac{TTE} is the sparsity of probe vehicle penetration. Several studies estimate a minimum of $2\%$ to $5\%$ for accurately determining traffic state and thereby inferring travel times.

Another crucial aspect in aggregating the speed of vehicles on roads is that vehicle speeds are inherently stochastic. We can notice vehicles in different lanes travel at different speeds (discounting various driver behaviours) depending upon the intended route. The randomness in road speeds is especially significant during peak hours as discussed in Section~\ref{subsec:speed-disribution}. Accurate lane level map matching is very challenging thus making it even harder to infer travel times. Finally, we also need to take into account dynamic changes to traffic conditions due to events such as accidents, road closures and thunder storms making accurate \ac{TTE}, a hard problem to solve. 

Approaches for TTE are broadly categorized as \emph{individual} and \emph{collective} based on the input data used in estimating TTE. In the individual approach, a given route is first split into several road  segments for which a travel time estimate is accurately calculated using (either pre-calculated or real-time or both) speed profiles of individual road segments. The final TTE is given simply as the sum of the travel times of the constituent segments. Earlier works such as~\cite{de2008traffic} and~\cite{nanthawichit2003application} fall in this category.
%

While collective methods can estimate individual travel times (via speed profiles) accurately, they ignore complexities arising due to road intersections, traffic lights, and turns. Hence, they additionally need  explicit models to capture time spent in the aforementioned overheads. However, individual methods appear to be more elegant while building large scale systems that should incorporate both historical and real time data.  
%
On the other hand, collective approaches  directly estimate the travel time for the entire path by implicitly considering inputs including the path length, time and day. Collective approaches are typically easier to build as they can be agnostic to the underlying routing constraints. However, their accuracy will vastly depend on the quality and coverage of historical data. The work in~\cite{jenelius2013travel} is an example of a collective method where the authors use \ac{FCD} to develop network models for estimating intersection delay and estimate spatio-temporal correlation between neighbouring links.


In this paper, we present a hybrid approach to \ac{TTE} that combines the advantages of both \emph{individual} and \emph{collective} methods. Our method operates on two phases: in the first phase, we compute an \emph{initial TTE} for a given origin-destination (OD) pair by relying on (historical) road speed profiles of the constituent road segments. However, since this initial TTE does not account for time delays encountered in traffic lights, intersections and turns,  in the second phase, we use a collective approach, where a \emph{final TTE} is predicted using a corrective machine learning (ML) model that takes the initial \ac{TTE}, along with time and day features as input. 

The first step in deriving an initial TTE is modeling \ac{RSP} for every road segment in the city's road network using our GPS probe data -- we use \ac{OSM} definition of road segments for our modeling purpose here. Road speed profile can be roughly defined as the expected speed of a vehicle for a particular road segment at given time and day. While we use \ac{RSP} only as intermediate result here, they are also widely used to detect abnormalities such as speeding cars, and traffic events~\cite{asakura2015incident}. 

By their very nature, \ac{RSP} exhibit huge variations owing to changes in vehicle speeds on the same segment over time, and the variety of GPS probe devices used to collect our observations (most of the ride-hailing cabs in Southeast Asia use cheap mobile phones with poor GPS receivers) An important problem here is to define the time intervals on which RSPs should be defined. Whereas most common way to capture historical speed profiles is by accumulating observations over pre-defined time buckets, they may not yield the best results, a simple \emph{one size fits all}  bucket may not be suitable for every time of the day. For instance, speed profiles during peak hours tend to have larger variations requiring finer time buckets, while non-peak hours tend to have very few rides, hence needing longer duration for reliable \ac{RSP}. To address this issue, we propose a regression tree based dynamic time buckets that adjust the time intervals based on the number of rides instead of just time.    

More precisely, the contributions of this paper are multi-fold: 
\begin{enumerate}
	\item We present a simple, yet effective approach to derive \ac{RSP} using \ac{FCD} from ride hailing cabs. 
	\item In order to capture speed profiles from historical data both in terms of their variance and coverage during different time intervals of the day and week, we propose an to use a decision tree based adaptive time buckets (detailed in section~\ref{subsec:adaptive-buckets}).
	\item We propose to employ a corrective \ac{ML} algorithm to implicitly model delays due to traffic lights, intersections, turn and other traffic events. The initial TTE from speed profiles will form an input to the corrective ML model. 
	\item Finally, we illustrate the effectiveness of our approach using a millions of rides done over a month in two mega-cities of Southeast Asia. To the best of our, this is one of the first attempts to study traffic patterns and travel times in S.E Asia using such as large scale data set.    
\end{enumerate}

The reminder of this paper is organized as follows: After providing the definitions for \ac{RSP} in the following section, we detail our approach to model \ac{RSP} in Section~\ref{sec:road-speed-profiling-details}.
%we discuss in detail the work-flow for constructing the \ac{RSP}. Specifically we talk about identifying pickup and drop off locations of a cab ride, . distribution of vehicle speeds and finally providing an \ac{EMA} based \ac{RSP} for each road segment on the road network graph. %
In Section~\ref{sec:travel-time-estimation} we discuss how we use the initial output of \ac{RSP} to give a final \ac{TTE} using a corrective ML model. 
In Section~\ref{sec:experiments}, we provide extensive experimental results of our approach on trips made in two mega cities of S.E Asia, hereafter referred to as City~$1$ and City~$2$ \footnote{We are avoiding specifying names of the cities deliberately.} for both pickup and drop off phases of a cab ride. The paper concludes in Section~\ref{sec:conclusion}. 


