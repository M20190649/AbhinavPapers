\section{Related work}
\label{sec:related-work}
 Using \ac{FCD} as a means for estimating travel times and traffic state has received much attention over the years. Papers such as~\cite{de2008traffic} and~\cite{nanthawichit2003application} compute the average travel time on roads from the speed of the probe vehicles. Artificial neural networks were used for short term predictions by using weighted aggregates of the past patterns in~\cite{de2008traffic}.  Using \ac{FCD} to detect events such as congestion and accidents has been discussed in~\cite{asakura2015incident}.  
 
 The percentage of probe vehicles required for reliable \ac{TTE} has been discussed in papers such as~\cite{dai2003simulation,hong2007spatial,kerner2005traffic}. Based on numerical simulations the authors of ~\cite{kerner2005traffic} conclude that a probe vehicle penetration of $1.5\%$ is required for reliable \ac{TTE}. 
 
 While the authors of~\cite{hong2007spatial} estimate the  probe penetration (to be around $2\%$) and the sampling period to maximize traffic information by determining {\it traffic time and space correlations}. Two spots are correlated in space if the correlation of average speeds in both the spots exceeds a threshold. Thus, the traffic condition in a spot without measurements can be estimated if the condition in a correlated spot is known. Likewise the correlation time measures the period over which the traffic conditions remain correlated in a spot hence influencing the sampling period. The traffic time and space correlations were measured using microscopic traffic simulations. Similarly~\cite{dai2003simulation} define confidence intervals for accuracy in terms of average speed and travel times across the links apart from spatial coverage as a function of probe penetration. The authors conclude that $3\%$ to $5\%$ probe penetration is sufficient for confidence levels of $90\%$ and above.
 
 In our work we do not have the means to estimate the percentage of probe vehicle penetration. While we are able to leverage on big data to process millions of driver trajectories to extract \ac{RSP}s followed by a corrective \ac{ML} algorithm for travel time estimation. 